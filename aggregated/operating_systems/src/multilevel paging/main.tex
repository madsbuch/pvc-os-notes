\documentclass[12pt]{article}
\usepackage[danish]{babel}
%\usepackage [latin1] {inputenc}
\usepackage [utf8] {inputenc}
\usepackage{moreverb}
\usepackage{listings}
\usepackage{graphicx}
\usepackage{geometry}
\usepackage{verbatim}
\usepackage{fancyref}
\usepackage{color}
\usepackage[usenames,dvipsnames]{xcolor}
\usepackage{amsmath}
\usepackage{centernot}
%\usepackage{tabularx}
\usepackage{ wasysym }
\usepackage{hyperref}
\usepackage{amsfonts}
\usepackage{ amssymb }
\usepackage{changepage}
\usepackage{lipsum}

\usepackage{natded}
\newcommand{\Lagr}{\mathcal{L}}

\usepackage{mathtools}
\DeclarePairedDelimiter{\ceil}{\lceil}{\rceil}
\DeclarePairedDelimiter{\floor}{\lfloor}{\rfloor}
\DeclarePairedDelimiter{\angleb}{\langle}{\rangle}

\definecolor{mygreen}{rgb}{0,0.6,0}
\definecolor{mygray}{rgb}{0.5,0.5,0.5}
\definecolor{mymauve}{rgb}{0.58,0,0.82}

\lstset{ %
  backgroundcolor=\color{white},   % choose the background color; you must add \usepackage{color} or \usepackage{xcolor}
  basicstyle=\footnotesize,        % the size of the fonts that are used for the code
  breakatwhitespace=false,         % sets if automatic breaks should only happen at whitespace
  breaklines=true,                 % sets automatic line breaking
  captionpos=b,                    % sets the caption-position to bottom
  commentstyle=\color{mygreen},    % comment style
  deletekeywords={...},            % if you want to delete keywords from the given language
  escapeinside={\%*}{*)},          % if you want to add LaTeX within your code
  extendedchars=true,              % lets you use non-ASCII characters; for 8-bits encodings only, does not work with UTF-8
  frame=single,                    % adds a frame around the code
  keepspaces=true,                 % keeps spaces in text, useful for keeping indentation of code (possibly needs columns=flexible)
  keywordstyle=\color{blue},       % keyword style
  language=Octave,                 % the language of the code
  mathescape=true,
  morekeywords={*,...},            % if you want to add more keywords to the set
  numbers=left,                    % where to put the line-numbers; possible values are (none, left, right)
  numbersep=5pt,                   % how far the line-numbers are from the code
  numberstyle=\tiny\color{mygray}, % the style that is used for the line-numbers
  rulecolor=\color{black},         % if not set, the frame-color may be changed on line-breaks within not-black text (e.g. comments (green here))
  showspaces=false,                % show spaces everywhere adding particular underscores; it overrides 'showstringspaces'
  showstringspaces=false,          % underline spaces within strings only
  showtabs=false,                  % show tabs within strings adding particular underscores
  stepnumber=1,                    % the step between two line-numbers. If it's 1, each line will be numbered
  stringstyle=\color{mymauve},     % string literal style
  tabsize=2,                       % sets default tabsize to 2 spaces
  title=\lstname                   % show the filename of files included with \lstinputlisting; also try caption instead of title
}


\newcommand{\defint}[4]{\int_{#1}^{#2} \! #3 \, \mathrm{d}#4}
\newcommand{\facit}[1]{\underline{\underline{#1}}}
\newcommand{\ssum}[2]{\sum\limits_{#1}^{#2}}

\usepackage{tikz}
\usepackage{amsmath, fancyhdr}
\usepackage{amsfonts}
\usepackage{amssymb}

\DeclareGraphicsExtensions{.pdf,.png,.jpg}
\begin{document}

\noindent
\subsection*{Multilevel Page Tables}
problem: Can't hold all tables in memory. \\
solution: Page the page tables. \\
For a 32 bit machine with the max of 4 GB of memory, if the pages are 4KB in size we need 1 million pages, each page table is 32 bits so 4MB. But every program needs its own page table with its own translations and they cannot be stored to disk. \\
What we do is create one 4KB table with indexes to 1024 other page tables. 
\subsection*{Translation proccess}
On IA-32 machines.
\begin{enumerate}
\item Divide the 32 bit address into the 20 left hand side bits for page indexing and the 12 right hand side bits for the offset within the actual page.
\item If the 20 bits is a TLB (Translation Buffer) hit then, then concatenate the result with the 12 bit offset And you are done.
\item Else divide the 20 bits into 2 10 bits page indexes, the primary which is the page directory which contains mappings to other page tables, and the secondary which points physical locations in memory.
\item Go to the value of the 10 first bits in the page directory, if it contains the address of a valid table (its valid bit is set), load that page table into memory, else you've got a page fault.
\item If the index of the last 10 bits of the first 20 bits is valid then you've found the corrects place in memory with the last 12 bits as its index, else this is a page fault.
\end{enumerate}
This principle can be extended to an arbitrary number of levels of page tables, similar to a shallow tree where each node has a lot of children. \\
Multilevel page tables can save a lot of memory - the only thing that has to be stored in main memory is the 4KB page directory. The downside is that multilevel page tables has a lot of overhead with multiple disc operation. 
\end{document}
